\documentclass[12pt]{article}


%%%%%%%%%%%%%%%%%%%%%%%%%
\pagestyle{empty}
\voffset = -8mm
\textheight = 225mm 
\hoffset = -14mm 
\textwidth = 155mm 
\evensidemargin =+1.5cm  
\oddsidemargin  =+1.5cm
\parindent8mm
\frenchspacing
%%%%%%%%%%%%%%%%%%%%%%%%%




\begin{document}

\centerline{\large{\bf{WEEK-END di CALCOLO delle VARIAZIONI}}}
\vspace*{0.035truein}
\centerline{\large{\bf WORKSHOP a FIRENZE}}
\vspace*{0.035truein}
\centerline{\small{\bf Programma}}
%\vspace*{0.035truein}
%\centerline{\small{\today}}
\vspace*{0.35truein}
%
%

\begin{itemize}

\centerline{\bf Venerd\`i 4 novembre 2005}

\item{ore 15.00: {\bf Cellina Arrigo} (Universit\`a di Milano-Bicocca),} 
\emph{``Sull'esistenza delle variazioni''}.

\item{ore 15.30: {\bf Bertone Simone} (Universit\`a di Milano-Bicocca),}
\emph{``On Hopf's Lemma and the Strong Maximum Principle''}.

\item{ore 16.00: {\bf Celada Pietro} (Universit\`a di Parma),} 
\emph{``Un modello variazionale policonvesso per le cavitazioni''}.

\item{ore 16.30-17.00: {\bf Pausa Caff\`e}.}

\item{ore 17.00: {\bf Treu Giulia} (Universit\`a di Padova),} 
\emph{``Lipschitz regularity of minima under slow growth assumptions''}.

\item{ore 17.30: {\bf Crippa Gianluca} (SNS Pisa),}
\emph{``Soluzioni oscillanti di equazioni di trasporto''}.


\vspace*{0.2truein}



\centerline{\bf Sabato 5 novembre 2005}



\item{ore 9.30: {\bf Leonetti Francesco} (Universit\`a di l'Aquila),}
\emph{``Stime puntuali per i minimi di alcuni funzionali vettoriali''}.

\item{ore 10.00: {\bf Mingione Giuseppe Rosario} (Universit\`a di Parma),}
\emph{``Stime Calderon-Zygmund per sistemi parabolici degeneri''}.

\item{ore 10.30: {\bf Passarelli Antonia} 
(Universit\`a di Napoli `Federico II'),}
\emph{``Regolarit\`a parziale per minimi locali di integrali 
quasiconvessi e policonvessi''}.


\item{ore 11.00-11.30: {\bf Pausa Caff\`e}.}

\item{ore 11.30: {\bf Gelli Maria Stella} (Universit\`a di Pisa),}
\emph{"Obstacle problem for free-discontinuity energies:
the periodically perforated case"}.

\item{ore 12.00: {\bf Zeppieri Caterina}, (Universit\`a `La Sapienza' Roma)}
\emph{``Un risultato di equi-integrabilit\`a nell'ambito delle 
teorie di riduzione di dimensione''}.


\item{ore 12.30-16.00: {\bf Pranzo Sociale}.} 


\item{ore 16.00: {\bf Marcelli Cristina} 
(Universit\`a Politecnica delle Marche),}
\emph{``Optimality conditions for non-convex, non-coercive
autonomous variational problems with constraints''}.


\item{ore 16.30: {\bf Petricca Pier Vincenzo} (Universit\`a di l'Aquila),}
\emph{ ``Esistenza di so\-lu\-zioni per un problema vettoriale ellittico 
con dato misura''}.


\item{ore 17.00-17.30: {\bf Pausa}.} 

\item{ore 17.30: {\bf Dragoni Federica} (SNS Pisa),} 
\emph{``Hopf-Lax formula for Hamilton-Jacobi Cauchy problems with
semicontinuous data and Hormander-Hamiltonian''}.

\item{ore 18.00: {\bf Santambrogio Filippo} (SNS Pisa),}
\emph{``Asymptotics of optimal complian\-ce location problems''}.


\vspace*{0.2truein}


\centerline{\bf Domenica 6 novembre 2005}

\item{ore 9.30: {\bf De Cicco Virginia} 
(Dipart. Me.Mo.Mat. Univ. `La Sapienza' Roma),}
\emph{``Rilassamento su $BV$ per funzionali integrali con integrando 
discontinuo''}.

\item{ore 10.00: {\bf Esposito Luca} (Universit\`a di Salerno),}
\emph{``Una versione quantitativa della disuguaglianza di Polya Szego''}.

\item{ore 10.30-11.00: {\bf Pausa}.}

\item{ore 11.00: {\bf Verde Anna} (Universit\`a di Napoli `Federico II'),} 
\emph{``Semicontinuit\`a inferiore per funzionali policonvessi in SBV''}.

\item{ore 11.30: {\bf Fusco Nicola} (Universit\`a di Napoli `Federico II'),}
\emph{``Una versione sharp della disuguaglianza isoperimetrica quantitativa''}.







 

\end{itemize}



\end{document}





