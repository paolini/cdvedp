\documentclass[12pt]{article}


%%%%%%%%%%%%%%%%%%%%%%%%%
\pagestyle{empty}
\voffset = -8mm
\textheight = 225mm 
\hoffset = -14mm 
\textwidth = 155mm 
\evensidemargin =+1.5cm  
\oddsidemargin  =+1.5cm
\parindent8mm
\frenchspacing
%%%%%%%%%%%%%%%%%%%%%%%%%




\begin{document}

\centerline{\large{\bf{WEEK-END di LAVORO di}}}
\vspace*{0.035truein}
\centerline{\large{\bf CALCOLO delle VARIAZIONI}}
\vspace*{0.035truein}
\centerline{\small{\bf{lista abstracts}}}
\vspace*{0.035truein}
\centerline{\small{\today}}
\vspace*{0.25truein}
%
%

\begin{itemize}

\item{\bf Bertone Simone} (Universit\`a di Milano-Bicocca),
\emph{``On Hopf's Lemma and the Strong Maximum Principle''}.

Abstract: In this paper we consider Hopf's Lemma
and the Strong Maximum Principle for supersolutions to
$$\sum_{i=1}^N g_i(u_{x_i}^2)u_{x_ix_i}=0$$
under suitable hypotheses that allow $g_i$ to assume
value zero at zero.


\item{\bf Celada Pietro} (Universit\`a di Parma), 
\emph{``Un modello variazionale policonvesso per le cavitazioni''}.

Abstract: Alcuni materiali elastici, sottoposti a grandi deformazioni al bordo,
si deformano elasticamente sino alla comparsa di una frattura o cavit\`a
all'interno. E' questo il problema delle cavitazioni studiato da J. Ball
(Philos. Trans. Roy. Soc. A306 1982) e successivamente da Stuart,
Marcellini, Mueller e Spector e molti altri.
Lo scopo di questa ricerca \`e discutere le caratterisiche di un modello
con energia policonvessa e lineare nel determinante che dia conto del
fenomeno della cavitazione.


\item{\bf Cellina Arrigo} (Universit\`a di Milano-Bicocca), 
\emph{``Sull'esistenza delle variazioni''}.


\item{\bf Crippa Gianluca} (SNS Pisa),
\emph{``Soluzioni oscillanti di equazioni di trasporto''}.

Abstract: Sia $X$ uno spazio vettoriale topologico di funzioni definite
su ${\bf R}^n$. Diciamo che $X$ ha la ``propriet� di chiusura'' se vale
il fatto seguente: preso un qualsiasi campo vettoriale f
dipendente dal tempo, con componenti appartenenti a $X$ e
divergenza limitata, e preso un qualsiasi dato iniziale $u_0$
in $X$, allora esiste una funzione $u$ dipendente dal tempo ed
appartenente a $X$ che risolve nel senso delle distribuzioni
l'equazione del trasporto con campo vettoriale $f$ e dato
iniziale $u_0$. Recentemente, A. Bressan ha posto la
questione dell'esistenza di uno spazio $X$ che soddisfi la
propriet� di chiusura, che si immerga in $L^1_{loc}$ con
compattezza e che contenga le funzioni limitate che siano
localmente in $BV$. In un lavoro in collaborazione con
Camillo De Lellis (Universit\`a di Zurigo) mostriamo che un
tale spazio $X$ non pu\`o esistere. La nostra prova si basa su
un esempio di N. Depauw, che mostra un'equazione di
trasporto malposta il cui campo vettoriale \`e ``quasi $BV$''.
Nel seminario indicher\`o le motivazioni che portano a
considerare spazi con le propriet\`a sopra elencate e, dopo
aver descritto l'esempio di Depauw, cercher\`o di mostrare la
strategia con cui procediamo nella nostra costruzione.


\item{\bf De Cicco Virginia} (Dipart. Me.Mo.Mat. Univ. `La Sapienza' Roma), 
\emph{``Rilassamento su $BV$ per funzionali integrali con integrando 
discontinuo''}.

Abstract: Si presentano alcuni risultati di semicontinuit\`a e 
rilassamento sullo spazio $BV$ per funzionali non coercivi e con 
crescita lineare, sotto deboli condizioni di regolarit\`a 
sull'integrando.


\item{\bf Dragoni Federica} (SNS Pisa), 
\emph{``Hopf-Lax formula for Hamilton-Jacobi Cauchy problems with
semicontinuous data and Hormander-Hamiltonian''}.

Abstract: In the first part I'll give some basic notions about
sub-Riemannian distances and viscosity solutions. Moreover
I'll quote some known  results of existence and uniqueness
for evolutive Hamilton-Jacobi equations, introducing the
Hopf-Lax formula.
In the second part I'll prove an existence result theorem
in the context of semicontinuos initial data and
Hormander-Hamiltonian. The key to prove this result is to
build a suitable type-distance solution for the associated
eikonal equation.
In the third part I'll prove a convergence theorem which
generelizes a known result for the usual inf-convolution.
In this part I'll use a Large Deviation Principle in the
hypoelliptic case, proved using basic notions of measure


\item{\bf Esposito Luca} (Universit\`a di Salerno),
\emph{``Una versione quantitativa della disuguaglianza di Polya Szego''}.

Abstract: Si prova una stima della differenza in norma $L^1$ tra una
funzione di Sobolev $u$ e la sua simmetrizzata radiale, in termini della
differenza dei rispettivi integrali di Dirichlet. Si giunge cos\`i a
stimare quanto sia ``vicina'' all'essere radiale (in senso $L^1$) una
funzione che verifichi ``quasi'' l'uguaglianza nella disuguaglianza di
Polya Szego.


\item{\bf Fusco Nicola} (Universit\`a di Napoli `Federico II'),
\emph{``Una versione sharp della disuguaglianza isoperimetrica quantitativa''}.

Abstract: Sia $E$ un insieme di perimetro finito, con $|E|=|B|$, dove $B$ \`e
la palla unitaria di $R^n$. Si prova che, a meno di traslazioni,
$$
|E\Delta B|\leq c(n)[D(E)]^{1/2},
$$
dove $c(n)$ \`e una costante che dipende solo dalla dimensione e
$$
D(E)={{P(E)-P(B)}\over{P(B)}}
$$
\`e il deficit isoperimetrico di $E$.


\item{\bf Gelli Maria Stella} (Universit\`a di Pisa),
\emph{"Obstacle problem for free-discontinuity energies:
the periodically perforated case"}.

Abstract: I will present some recent results regarding Gamma-limits
for Mumford-Shah type energies in presence of unilateral
constraint on periodically perforated sets. 
Up to select the (unique) meaningful scaling threshold, in the limit, 
a penalization term appears, taking into account the 1-capacity of
(a regularization of) the reference set. 
The result is proved also for the case of reference 
perforations that are Lebesgue negligible. 
In such a setting the constraint must be expressed 
by means of a suitable $H^{n-1}$-representant of 
a $BV$ function.



\item{\bf Leonetti Francesco} (Universit\`a di l'Aquila),
\emph{``Stime puntuali per i minimi di alcuni funzionali vettoriali''}.

Abstract: Consideriamo il funzionale
$$
I(u) = \int_{\Omega} f(x,Du(x))dx
$$
definito su una classe opportuna di applicazioni vettoriali
$u:\Omega \subset{\bf R}^n \to{\bf R}^N$.

Presenteremo due condizioni sull'integrando $f$
che garantiscano la validit\`a del principio
del massimo per le componenti $u^j$ dei minimi del
funzionale $I$: la prima assicura che
tra i minimi di I  ce n'\`e uno che verifica
il principio del massimo; la seconda garantisce che tutti
i minimi lo verificano. Osserviamo che f non \`e supposta
strettamente convessa, dunque ci pu\`o essere pi\`u di un minimo.

Alcuni integrali $I$ hanno problemi di semicontinuit\`a inferiore;
per ovviare a questo problema, consideriamo il funzionale
rilassato
$$
RI(u) = \inf\{\liminf_k I(u_k): u_k \to u\} 
$$
e diamo condizioni per la validit\`a del principio del massimo 
per i minimi del funzionale rilassato.



\item{\bf Marcelli Cristina} (Universit\`a Politecnica delle Marche),
\emph{``Optimality conditions for non-convex, non-coercive
autonomous variational problems with constraints''}.

Abstract:  We consider the classical autonomous constrained
variational problem of minimization of  $ \int_a^b f(v(t),v'(t))
dt$ in the class \  $\Omega:=\{v \in W^{1,1}(a,b): v(a)=\alpha,
v(b)= \beta, v'(t)\ge 0 \ \mbox{a.e. in } (a,b) \}$, where 
$f:[\alpha, \beta]\times [0,+\infty) \to {\bf R}$ is a lower
semicontinuous, non-negative integrand, which can be nonsmooth,
nonconvex, noncoercive.

We prove a necessary and sufficient condition for the optimality
of a trajectory \ $v_0\in \Omega$ in the form of a DuBois-Reymond
inclusion involving the subdifferential of Convex Analysis.
Moreover, we also provide a relaxation result and necessary and
sufficient conditions for the existence of the minimum expressed
in terms of an upper limitation for the assigned mean slope \
$\xi_0=(\beta-\alpha)/(b-a)$. Applications to various noncoercive
variational problems are also included.


\item{\bf Mingione Giuseppe Rosario} (Universit\`a di Parma),
\emph{``Stime Calderon-Zygmund per sistemi parabolici degeneri''}.

Abstract: Le stime tipo Calderon-Zygmund permettono di ottenere, per
equazioni e sistemi non omogenei, informazioni sull'integrabilit\'a
della soluzione a partire da quella del "dato". Sono un classico nel
caso lineare ellittico e parabolico. Per il $p$-Laplaciano sono invece
una conquista piu' recente di T. Iwaniec (equazioni), e
DiBenedetto-Manfredi (sistemi). Le tecniche note, basate dull'uso di
opportuni strumenti di analisi armonica (commutatori, funzione
massimale sharp)e stime $C^{1,\alpha}$, non si applicano al caso
evolutivo, che \`e rimasto aperto. Presento un risultato, ottenuto con
Emilio Acerbi, in cui questo gap viene infine colmato. La
dimostrazione non fa uso di tecniche di Analisi Armonica, e si fonda
su stime $C^{0,1}$, e non pi\'u $C^{1,\alpha}$. 
Essa si applica ovviamente al caso ellittico, e permette inoltre di 
ritrovare i risultati per equazioni a coefficienti discontinui 
(alla Chiarenza-Fracs-Longo), in modo completamente elementare. 
Il risultato in questione \`e infine: sia $u$ la soluzione di
$$
u_t-div(|Du|^{p-2}Du)=div(|F|^{p-2}F),
$$
con $p>1$, allora
$$
F \in L^{q}_{loc} \Rightarrow Du \in L^{q}_{loc}
$$
$\forall q \geq p$.


\item{\bf Passarelli Antonia} (Universit\`a di Napoli `Federico II'), 
\emph{``Regolarit\`a parziale per minimi locali
di integrali quasiconvessi e policonvessi''}.

Abstract: Esporremo alcuni risultati concernenti la regolarit\'a parziale
$C^{1,\alpha}$ per i minimi locali di funzionali integrali del tipo
$$ 
I(v)=\int_{\Omega }F(Dv(x))dx
$$
dove l'integrando $F(\xi)$ ha crescita subquadratica, cio\'e 
$|F(\xi)|\leq L(1+|\xi|^p)$, con $ 1<p<2$, e funzionali 
policonvessi del tipo
$$ 
I(u)=\int_\Omega|Du|^2+f({\rm Adj}Du)+g({\rm det}Du),
$$
dove $u:\Omega\subset{\bf R}^3\to{\bf R}^3$, $f$ cresce come 
$|{\rm Adj}Du|^p$, $g$ come $|{\rm det}Du|^q$ e $1<q<p<2$.


\item{\bf Petricca Pier Vincenzo} (Universit\`a di l'Aquila),
\emph{ ``Esistenza di soluzioni per un problema vettoriale ellittico 
con dato misura''}.

Abstract: Consideriamo il problema di Dirichlet
$$\left\{\begin{array}{ll}
-\sum_{i=1}^n D_i ( |D_i u|^{p_i-2}D_i u ) = \mu &\textrm{in\,} \Omega\\
u=0  &\textrm{su\,} \partial \Omega\end{array}\right.
$$
dove
$\Omega$ \`e un aperto limitato di ${\bf R}^n$ e
$\mu$ \`e una assegnata misura vettoriale di Radon finita su 
${\bf R}^n$ a valori in ${\bf R}^N$. 
Sotto opportune ipotesi sugli esponenti $p_i \geq 2$ si mostra l'esistenza
di una soluzione $u:\Omega\to{\bf R}^N$.

Questo problema si inquadra nell'ambito dei sistemi ellittici con crescita
anisotropa.


\item{\bf Santambrogio Filippo} (SNS Pisa), 
\emph{``Asymptotics of optimal compliance-location problems''}.

Abstract: We consider the problem of placing a
Dirichlet region made by $n$ small balls of given radius
in a given domain subject to a force $f$ in order to
minimize the compliance of the configuration. Then we let
$n$ tend to infinity and look for the $\Gamma$-limit of
suitably scaled functionals, in order to get informations
on the asymptotical distribution of the centres of the
balls (that will tend to concentrate mostly in the regions
where $f$ is stronger). This problem is both linked to
optimal location and shape optimization problems.


\item{\bf Treu Giulia} (Universit\`a di Padova),
\emph{``Lipschitz regularity of minima under slow growth assumptions''}.

 Abstract: The Lipschitz regularity of minima for one dimensional 
 variational problems is a classical problem in the framework of the 
 Calculus of Variations and is the crucial point to investigate more 
 regularity properties.

 The classical results in this direction make strongly use of the coercivity 
assumptions on the Lagrangian. We present some recent theorems in which the 
main point is a "geometric" growth assumption that include both the coercive 
functionals and a class of functions with linear growth.


\item{\bf Verde Anna} (Universit\`a di Napoli `Federico II'), 
\emph{``Semicontinuit\`a inferiore per funzionali policonvessi in SBV''}.

Abstract: Un risultato di semicontinuit\`a inferiore si prova per 
funzionali composti da una energia policonvessa ed un termine di 
superficie. Tale teorema estende al contesto delle funzioni $SBV$ 
un ben noto risultato di J. Ball del 1977.


\item{\bf Zeppieri Caterina} (Universit\`a `La Sapienza' Roma), 
\emph{``Un risultato di equi-integrabilit\`a nell'ambito delle 
teorie di riduzione di dimensione''}.
 
Abstract: Nell'ambito dei problemi di riduzione di dimensione
per materiali elastici nonlineari, si incontrano energie che dipendono
da successioni di 'gradienti riscalati', ovvero da successioni 
della forma 
$(\nabla_\alpha u_\varepsilon|\frac{1}{\varepsilon}\nabla_3u_\varepsilon)$
dove $x_3$ \`e la direzione 'sottile' e $\nabla_\alpha$ rappresenta la
derivazione nelle rimanenti variabili.

Un risultato di Bocea e Fonseca mostra che, sotto ipotesi di limitatezza
in\\ $L^p(\Omega,{\bf R}^{3\times 3})$, una tale successione pu\`o essere 
decomposta nella somma di una successione 
$(\nabla_\alpha v_\varepsilon|\frac{1}{\varepsilon}\nabla_3v_\varepsilon)$ 
la cui potenza $p$-esima \`e equi-integrabile in $\Omega$ pi\`u un resto
che converge a zero in misura.

Un tale lemma \`e di grande utilit\`a nella determinazione delle teorie
'ridotte' a partire da quelle tridimensionali, tramite il calcolo di un 
$\Gamma$-limite.

Diamo una dimostrazione alternativa di questo risultato basata su un metodo
di riscalamento che ci permette di utilizzare l'analogo risultato, 
per 'gradienti non riscalati', di Fonseca, M\"uller e Pedregal e di 
concludere tramite una semplice applicazione del Criterio di 
de la Vall\'ee Poussin.
\end{itemize}



\end{document}





